\section{GIT}

\url{https://git-scm.com/book/pl/v1}

\paragraph{git bisect\\}
Wyszukiwanie punktu pojawienia się błędu:
\begin{enumerate}
	\item \texttt{git bisect start} -- rozpocznij wyszukiwanie,
	\item \texttt{git bisect bad <coś tam>} -- podaj commit w którym występuje błąd,
	\item \texttt{git bisect good <coś tam>} -- podaj commit w którym błąd nie występuje,
	\item sprawdzaj stan i sygnalizuj go komendami \texttt{git bisect good} lub \texttt{git bisect bad} do momentu odnalezienia punktu pojawienia się błędu, git sam będzie się przełączał pomiędzy commitami,
	\item \texttt{git bisect reset} -- zakończ przeszukiwanie.
\end{enumerate}

\paragraph{git blame \texttt{plik}\\}
Pokazuje autorów zmian poszczególnych linii w pliku.

\paragraph{git branch\\}
Zarządzanie gałęziami, bez argumentów wyświetla listę.
\begin{itemize}
	\item git branch -d \texttt{nazwa} -- usuwa gałąź.
	\item git branch -v -- lista gałęzi z ostatnimi commitami.
	\item git branch - -merged -- lista gałęzi scalonych do aktywnej.
	\item git branch - -no-merged -- lista gałęzi niescalonych z aktywną.
\end{itemize}

\paragraph{git checkout \texttt{nazwa}\\}
Przełącza gałąź.
\begin{itemize}
	\item -b -- tworzy i przełącza się na nowo tworzoną gałąź.
\end{itemize}

\paragraph{git clone \texttt{adres}\\}
Kopiuje repozytorium z podanego adresu.

\paragraph{git config\\}
Zarządzanie parametrami.
\begin{itemize}
	\item -l -- wyświetla listę aktualnych parametrów.
	\item git config user.name ''XXX'' -- ustawianie użytkownika.
	\item git config user.email ''YYY'' -- ustawianie adresu użytkownika.
	\item git config - -bool core.bare true -- repozytorium bazowe.
	\item git config core.editor -- ustawia edytor tekstu.
\end{itemize}

\paragraph{git commit}
Zapisuje zmiany.
\begin{itemize}
	\item - -amend -- poprawka do poprzedniego commita.
\end{itemize}

\paragraph{git describe\\}
Tworzy tekst opisujący wersję względem ostatniego tagu.

\paragraph{git diff\\}
Wyświetla zmiany w obserwowanych plikach.
\begin{itemize}
	\item - -cached -- wyświetla zmiany w buforze względem ostatniego commita.
\end{itemize}

\paragraph{git fetch\\}
Pobiera dane ze źródła ale ich nie scala.

\paragraph{git help \texttt{polecenie}\\}
Wyświetla pomoc do polecenia.

\paragraph{git init}
Tworzy nowe, puste repozytorium gita.
\begin{itemize}
	\item - -bare -- repozytorium bazowe.
\end{itemize}

\paragraph{git log\\}
Wyświetla historię.
\begin{itemize}
	\item - -oneline
	\item - -graph
	\item - -decorate
	\item - -all
\end{itemize}

\paragraph{git merge \texttt{gałąź}\\}
Scala wskazaną gałąź z aktywną gałęzią.

\paragraph{git mergetool\\}
Uruchamia narzędzie do scalania.
\begin{itemize}
	\item - -tool=\texttt{narzędzie} -- wskazuje konkretne narzędzie do scalania.
\end{itemize}

\paragraph{git pull \texttt{index} \texttt{branch}\\}
Próbuje automatycznie pobrać dane ze źródła i scalić z aktualną gałęzią.

\paragraph{git push \texttt{index} \texttt{branch}\\}
Próbuje wysłać aktualną gałąź do repozytorium źródłowego.
\begin{itemize}
	\item - -tags -- wysyła listę tagów.
\end{itemize}

\paragraph{git reflog\\}
Wyświetla logi gita.

\paragraph{git remote\\}
Ustawienia repozytorium źródłowego.
\begin{itemize}
	\item -v -- wyświetla listę.
	\item git remote show \texttt{indeks} -- wyświetla dane o wskazanym źródle.
	\item git remote add \texttt{indeks} \texttt{adres} -- dodaje źródło.
\end{itemize}

\paragraph{git reset --hard \texttt{commit\_nr}\\}
Przełączenie wskaźnika HEAD na dowolny commit.

\paragraph{git rm --cached \texttt{plik}\\}
Usuwa wskazany plik z repozytorium.

\paragraph{git show \texttt{commit\_nr} / \texttt{commit} / \texttt{etykieta}\\}
Wyświetla informacje o wskazanym obiekcie.

\paragraph{git show-branch\\}
Wyświetla listę gałęzi.
\begin{itemize}
	\item - -all
	\item - -list
\end{itemize}

\paragraph{git stash\\}
Zachowuje aktualne zmiany na stosie, moży być wykorzystany do przenoszenia zmian pomiędzy gałęziami.
\begin{itemize}
	\item \texttt{list} -- wyświetla listę zachowanych zmian,
	\item \texttt{pop} -- wczytuje zachowane zmiany ze stosu,
	\item \texttt{drop} -- porzuca zmiany zachowane na stosie.
\end{itemize}

\paragraph{git show <arg>\\}
Opisuje wskazany obiekt, bez argumentu opisze ostatni commit.

\paragraph{git shortlog\\}
Wyświetla listę commitów dla poszczególnych autorów.
\begin{itemize}
	\item \texttt{-s} -- wyświetla tylko liczbę commitów danego autora,
	\item \texttt{-n} -- sortuje autorów według liczby commitów.
\end{itemize}

\paragraph{git tag\\}
Wyświetla listę tagów.
\begin{itemize}
	\item -a \texttt{etykieta} -m''\texttt{opis}'' -- dodaje tag z opisem.
	\item - -list
	\item  - -delete \texttt{etykieta}
	\item git tag \texttt{etykieta} -- dodaje prosty tag.
\end{itemize}

